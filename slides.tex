\documentclass[xcolor=table]{beamer}

\useinnertheme{metropolis}
\useoutertheme{metropolis}
\usecolortheme{metropolis}
\usefonttheme{professionalfonts}         % Use metropolis theme

\usepackage{fontspec}
\setsansfont{Fira Sans Light}
\setmonofont{Fira Mono}
\setmainfont{Fira Sans Light}

\usepackage[bibstyle=unified, citestyle=unified, backend=biber]{biblatex}
\addbibresource{fdsl.bib}

\usepackage{longtable}
\usepackage{tipa}
\usepackage{booktabs}
\usepackage{tikz}
\usepackage{tikz-qtree}
\usepackage{expex}
\usepackage[normalem]{ulem}
\usepackage{xcolor}
\usepackage{float}
\usepackage{etoolbox}
\usepackage{multirow}
\usepackage{graphicx}
\usepackage{wrapfig}
\usepackage{subfig}
\usepackage{mathabx}
\usepackage{hyperref}

\usetikzlibrary{matrix}
\usetikzlibrary{positioning}
\usetikzlibrary{arrows.meta}
\usetikzlibrary{tikzmark}
\usetikzlibrary{decorations.shapes}

\tikzset{mytree/.style={baseline=(top.base),
level distance=2.5em, sibling distance=4em, align=center,
parent anchor=south, child anchor=south, anchor=north}}

	\hypersetup{
		colorlinks=true,
		linkcolor=black,
		citecolor=magenta,
		filecolor=black,
    		urlcolor=black,
	}

\lingset{numoffset=1ex, belowglpreambleskip=0ex, aboveglftskip=0ex, belowexskip=2ex, aboveexskip=2ex} % gloss formatting


\renewcommand*{\bibfont}{\small}
\setbeamertemplate{bibliography item}{}


\AtBeginSection[]{
  \begin{frame}
  \vfill
  \centering
  \usebeamerfont{title}\insertsectionhead\par%
  \vfill
  \end{frame}
}


\title{F=PL синкретизм и дефолтный род в шугнанском языке}
\date{ТМП 13.10.2022}
\author[shortname]{Данияр Касенов \inst{1} \and Александр Сергиенко \inst{2} \and Артём Бадеев \inst{3}}
\institute[shortinst]{\inst{1},\inst{2} НУЛ по формальным моделям в лингвистике ВШЭ, МГУ \and \inst{3} ИКВИА ВШЭ}

\begin{document}

\maketitle

\section{Введение}

\begin{frame}{О докладе}

	\begin{itemize}
		\item Цель доклада -- дать анализ синкретизму F.SG=M/F.PL в согласовании в шугнанском языке
		\item Также будет показано, чем шугнанский отличается от других языков с таким синкретизмом
		\item Исследование поддержано грантом РФФИ №20-512-26004 ``Морфология согласования"
	\end{itemize}

\end{frame}

\begin{frame}{Проблема}

\begin{table}[h]
\centering
\caption{Спряжение глагола \textit{sitow} `остаться' в перфекте}
\begin{tabular}{lll}
   & M                         & F                         \\
SG & suδǰ                        & \cellcolor[HTML]{FCFF2F}sic \\
PL & \cellcolor[HTML]{FFFE65}sic & \cellcolor[HTML]{FCFF2F}sic
\end{tabular}
\end{table}

\end{frame}

\begin{frame}{Проблема}

\begin{table}[h]
\centering
\caption{Спряжение глагола \textit{sitow} `остаться' в прошедшем}
\begin{tabular}{lll}
   & M                         & F                         \\
SG & sut                      & \cellcolor[HTML]{FCFF2F}sat \\
PL & \cellcolor[HTML]{FFFE65}sat & \cellcolor[HTML]{FCFF2F}sat
\end{tabular}
\end{table}

\end{frame}

\begin{frame}{Проблема}
\begin{table}[h]
\centering
\caption{Спряжение глагола \textit{vidow} `быть' в перфекте}
\begin{tabular}{lll}
   & M                         & F                         \\
SG & vuδǰ                        & \cellcolor[HTML]{FCFF2F}vic \\
PL & \cellcolor[HTML]{FFFE65}vic & \cellcolor[HTML]{FCFF2F}vic
\end{tabular}
\end{table}

\end{frame}

\begin{frame}{Проблема}

\begin{table}[h]
\centering
\caption{Спряжение глагола \textit{vidow} `быть' в прошедшем}
\begin{tabular}{lll}
   & M                         & F                         \\
SG & vud                     & \cellcolor[HTML]{FCFF2F}vad \\
PL & \cellcolor[HTML]{FFFE65}vad & \cellcolor[HTML]{FCFF2F}vad
\end{tabular}
\end{table}

\end{frame}

\begin{frame}{Проблема}
	
	Вывод: форма M.SG самая маркированная (что необычно!)

	Синкретизм F=PL
	
\end{frame}

\begin{frame}{Дальше}

	(a) Теоретические подходы к синкретизму в дистрибутивной морфологии

	(б) Синкретизм F=PL не в шугнанском

\end{frame}

\section{Теоретические подходы к синкретизму}

\begin{frame}{DM: введение}

	\begin{itemize}
		\item Реализационная модель морфологии: синтаксическая структура (полностью абстрактная) превращается в морфонологические реперезентации путём межмодулярного перевода (inter-modular translation)

		\item Реализация структуры идёт по терминалам в контекстах

		\item Имеются морфологически-специфичные операции над признаками
	\end{itemize}

\end{frame}

\begin{frame}{DM: пример анализа}

	Спряжение в английском: I/we/you/they walk, he/she/it walk-s 

	\pex Правила лексической вставки
		\a T $\leftrightarrow$ /-s/ /\_\_Agr[3,\textsc{sg}]
		\a T $\leftrightarrow$ /-$\emptyset$/
	\xe

\end{frame}

\begin{frame}{Как делать синкретизм в DM}

	Вариант 1: недоспецификация в правилах (сделать так, чтобы одно правило подходило в нескольких случаях)

	Иногда это не подходит: если синкретизм систематический

\end{frame}

\begin{frame}{Синкретизм согласования в немецком}

	\begin{table}[h]
\centering
\caption{Спряжение в немецком глагола \textit{trinken} `пить'}
\begin{tabular}{lll}
   & Sg                        & Pl                         \\
1 & trink-e               & \cellcolor[HTML]{FCFF2F}trink-en \\
2 & trink-st & trink-et \\
3 & trink-t & \cellcolor[HTML]{FCFF2F}trink-en
\end{tabular}
\end{table}

\end{frame}

\begin{frame}{Синкретизм согласования в немецком}

	\begin{table}[h]
\centering
\caption{Спряжение в немецком глагола \textit{trinken} `пить'}
\begin{tabular}{lll}
   & Sg                        & Pl                         \\
1 & bin         & \cellcolor[HTML]{FCFF2F}sind \\
2 & bist & seid \\
3 & ist & \cellcolor[HTML]{FCFF2F}sind
\end{tabular}
\end{table}

\end{frame}

\begin{frame}{Операция обеднения}

	Необходимо совпадение правил лексической вставки как для регулярных, так и иррегулярных глаголов

	Более общее решение: нейтрализация оппозиции 1vs3 в множественном числе

	Обеднение: [1]/[3] $\rightarrow$ $\emptyset$ /\_\_[+\textsc{pl}]

\end{frame}

\section{Синкретизм F=PL вне шугнанского}

\begin{frame}{Данные языка сидаама}

	\begin{table}[h]
\centering
\caption{Согласовательные маркеры в сидаама в перфекте}
\begin{tabular}{lll}
   & M                         & F                         \\
SG & -í                & \cellcolor[HTML]{FCFF2F}-tú \\
PL & \cellcolor[HTML]{FFFE65}-tú & \cellcolor[HTML]{FCFF2F}-tú
\end{tabular}
\end{table}
	
\end{frame}

\begin{frame}{Анализ Kramer, Teferra 2019}

	(a): система двух родов $\Rightarrow$ [$\pm$\textsc{fem}]

	(b): механизм обеднения, удаляющего синтаксическую вершину (уничтожение)

	(b) необходимо, потому что просто удаление [\textsc{pl}] не подойдет для мужского рода в мн.ч.

\end{frame}

\begin{frame}{Анализ для сидаама}

	Уничтожение: Agr[\textsc{+pl}] $\rightarrow$ $\emptyset$ /\_\_Asp[\textsc{+pfv}]

	\pex Правила лексической вставки
		\a Asp[\textsc{pfv}] $\leftrightarrow$ /-í/ /\_\_Agr[3][\textsc{-fem}]
		\a Asp[\textsc{pfv}] $\leftrightarrow$ /-tú/
	\xe

	Следствие правил вставки: женский род появляется как дефолтный (при, например, отсутствии осмысленного Agr). Это предсказание подтверждается для сидаама, см. Kramer, Teferra 2019

\end{frame}

\section{Анализ шугнанского}

\begin{frame}{Сидаама не шугнанский}

	Можно было бы применить анализ Kramer, Teferra (2019) к шугнанскому

	Но: в шугнанском дефолтный род не женский, а мужской!

\end{frame}

\begin{frame}{Согласование с сентенциальными актантами}

	\ex
		\begingl
			\gla paxta δīvd 		ɣal tajor na-su-δǰ//
			\glb хлопок собирать.\textsc{inf} 	ещё конец \textsc{neg}-стать-\textsc{pf.m}//
			\glft ‘Сбор хлопка еще не закончился’ (Карамшоев 1963: 254)//
		\endgl
	\xe

\end{frame}

\begin{frame}{Пропозициональная анафора}

	\ex
		\begingl
			\gla Fuk-aθ        	di   fam-en //       
			\glb все-\textsc{adv}   \textsc{d2.m.sg.o} знать-\textsc{3pl}//
			\glft `Все это знают.’//
		\endgl
	\xe

\end{frame}

\begin{frame}{Анализ шугнанского}

	(a) Два бинарных признака [$\pm$\textsc{anim}][$\pm$\textsc{fem}]

	(a') M=[+\textsc{anim}][--\textsc{fem}]; F=[+\textsc{anim}][+\textsc{fem}]

	(b) Обеднение [+\textsc{anim}] $\rightarrow$ $\emptyset$ /\_\_[--\textsc{pl}][--\textsc{fem}]

\end{frame}

\begin{frame}{Лексическая вставка}

	\pex
		\a T[\textsc{pf}] $\leftrightarrow$ /-ic/ /\_\_Agr[+\textsc{anim}]\\
		\a T[\textsc{pf}] $\leftrightarrow$ /-uδǰ/\\
		\a T[\textsc{pst}]$\leftrightarrow$ /-a-/ /\_\_ Agr[+\textsc{anim}]\\
		\a T[\textsc{pst}] $\leftrightarrow$ /-u-/
	\xe

	Форма мужского рода ед.ч. дефолтная (elsewhere)

\end{frame}

\begin{frame}{Почему два признака для двух родов?}

	Диахрония: в праиранском было три рода (два бинарных признака)

	Следовательно, это диахронический остаток
\end{frame}

\begin{frame}{Почему именно такое обеднение}

	Мы предполагаем, что наше правило обеднения отражает процесс слияния мужского рода со средним в процессе становления шугнанского из праиранского

\end{frame}

\begin{frame}{Слияние родов}

	\begin{table}[]
\centering
\caption{Основы прилагательных в староавестийском языке [Beekes]}
\label{tab:my-table}
\begin{tabular}{llll}
\multicolumn{2}{c}{M}            & \multicolumn{2}{c}{N} \\
-a & 170 & -a       & 45         \\
-u & 19  & -u       & 1+130      \\
-i  & 4  & -i  & 1  \\
-nt & 20 & -nt & 10 \\
-n  & 35 &     &    \\
-h  & 2  & -h  & 1  \\
-C  & 7  &     &   
\end{tabular}
\end{table}

\end{frame}


\begin{frame}{Итого}

 Мы дали анализ удивительного синкретизма в шугнанском используя два диахронически мотивированных допущения
	
	Спасибо за внимание!
\end{frame}


\begin{frame}{Множественное число}

	Наш анализ предсказывает, что признаковая спецификация [\textsc{+pl}] (без [\textsc{+anim}]) будет реализовываться как \textsc{m.sg}

	Возможный такой контекст: сочиненные сентенциальные актанты

	\ex \begingl 
		\gla Chilim          tizhd-at sharob                    birekht=en salomate-ra ziyùn sat//
		\glb сигарета   курить-\textsc{add} алкоголь      пить=\textsc{3pl}   здоровье-\textsc{lat} вред стать.\textsc{pl}//
		\glft `Курить сигареты и пить алкоголь стало вредно'//
	\endgl \xe

	Проблема? Зависит от анализа согласования с сентенциальными актантами

\end{frame}

\end{document}